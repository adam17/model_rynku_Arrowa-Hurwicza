\documentclass[a4paper,11pt]{beamer}

\mode<presentation>
{
  \usetheme{Madrid}
  % or ...
  %\usetheme{Darmstadt}
  \usefonttheme[onlylarge]{structurebold}
  \setbeamerfont*{frametitle}{size=\normalsize,series=\bfseries}
  \setbeamertemplate{navigation symbols}{}


  \setbeamercovered{transparent}
  % or whatever (possibly just delete it)
}

\usepackage{polski}
\usepackage[utf8]{inputenc}
\usepackage{times}
\usepackage[T1]{fontenc}

\title[Model rynku Arrowa-Hurwicza]
{
  Równowaga rynkowa\\
  Model rynku Arrowa-Hurwicza
}

\author[Kruszewski, Paulukanis, Sochoń]
{
  Mateusz Kruszewski  \and
  Adam Paulukanis     \and
  Paweł Sochoń
}

\institute[UwB]
{
  Uniwersytet w Białymstoku
}

\date[Białystok, 2012]
{
  Białystok, 2012
}

\AtBeginSubsection[]
{
  \begin{frame}<beamer>{Outline}
    \tableofcontents[currentsection,currentsubsection]
  \end{frame}
}

\begin{document}

  \begin{frame}
    \titlepage
  \end{frame}

  \begin{frame}{Outline}
    \tableofcontents
  \end{frame}


  \section{Równowaga rynkowa}
    \subsection{Wstęp}
    \subsection{Model rynku wg. Arrowa-Hurwicza}
    \subsection{Model rynku wg. Arrowa-Hurwicza c.d.}
    \subsection{$k$-ty kupiec wybiera koszyk $x^k$}
    \subsection{Wektor nadmiernego (nadwyżkowego) popytu na towar}
    \subsection{Rynek w równowadze}
    \subsection{Jeden dodatni wektor cen równowagi rynkowej}

  \section{Przykład}
    \subsection{Przykład 1}

\end{document}
